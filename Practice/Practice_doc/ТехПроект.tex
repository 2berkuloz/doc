\section{Технический проект}
\subsection{Общая характеристика организации решения задачи}

Разрабатывается клиентская сторона веб-приложения для чат-платформы, обеспечивающая взаимодействие пользователя с серверной частью программы и базой данных. Клиентская сторона реализует следующие основные функции:
\begin{enumerate}
	\item Аутентификация и авторизация пользователей
	\item Обмен сообщениями в групповых чатах
	\item Личная переписка между пользователями
	\item Управление группами и правами участников
	\item Работа с вложениями и медиафайлами
\end{enumerate}

\subsection{Обоснование выбора технологии проектирования}

Для реализации клиентской стороны выбраны следующие технологии:

\subsubsection{Язык программирования JavaScript}

JavaScript выбран благодаря:
\begin{itemize}
	\item Поддержки в браузерах
	\item Простоте и читаемости кода
	\item Популярности и востребованности
	\item Кроссплатформенности
	\item Асинхронности и динамике
\end{itemize}

\subsubsection{Язык программирования CSS}

В качестве основы разметки используется CSS (Cascading Style Sheets) - основной язык стилизации, обеспечивающий гибкость и удобство работы с дизайном:
\begin{itemize}
	\item Легковесность - стили загружаются отдельно от HTML
	\item Простота развертывания - подключение к HTML через встроенные стили
	\item Совместимость - работает во всех современных браузерах
	\item Адаптивность - поддержка медиа-запросов для различных устройств
\end{itemize}

\subsubsection{Язык разметки HTML}

В качестве основы структуры веб-страницы используется HTML (HyperText Markup Language) - главный язык разметки, обеспечивающий логическую организацию контента и взаимодействие с CSS и JavaScript
\begin{itemize}
	\item Структурированность - позволяет логически организовывать элементы страницы
	\item Универсальность - работает во всех современных браузерах как для статических страниц, так и для динамических веб-приложений
	\item Кроссплатформенность - работает на любом устройстве (компьютерах, планшетах, смартфонах)
	\item Адаптивность - поддерживает медиа-запросы и гибкие макеты
\end{itemize}


\subsection{Клиентская часть мессенджера}


\subsubsection{Структура проекта}
\begin{xltabular}{\textwidth}{|l|X|}
	\caption{Файловая структура клиентской части}\label{tab:project-structure} \\ \hline
	\centrow{Тип} & \centrow{Описание} \\ \hline 
	\endfirsthead
	\continuecaption{Продолжение таблицы \ref{tab:project-structure}}
	\finishhead
	HTML-страницы & 
	\begin{itemize}[leftmargin=*,nosep]
		\item index.html — основной интерфейс мессенджера
		\item login.html — страница авторизации
		\item register.html — страница регистрации
		\item 404.html, 500.html — страницы ошибок
	\end{itemize} \\ \hline
	
	Стили & 
	\begin{itemize}[leftmargin=*,nosep]
		\item style.css — основные стили приложения
		\item login.css, register.css — стили форм авторизации
		\item error.css — стили страниц ошибок
	\end{itemize} \\ \hline
	
	JavaScript & 
	\begin{itemize}[leftmargin=*,nosep]
		\item app.js — основной клиентский код 
		\item login.js — обработка формы входа
		\item register.js — обработка формы регистрации
	\end{itemize} \\ \hline
	
\end{xltabular}

\subsection{Ключевые функциональные модули}

\subsubsection{Управление интерфейсом}
Система взаимодействия с пользователем реализуется через initUI(), инициализирующую все элементы при загрузке страницы. toggleSidebar() управляет видимостью боковых панелей, синхронизируя их состояние с SessionStorage. Динамическое обновление блока авторизации (updateAuthButtons()) отражает изменения статуса пользователя, а унифицированная система уведомлений (showToast()) использует анимированные всплывающие сообщения с цветовой индикацией типа события.

\subsubsection{Обработка сообщений}
Механизм работы с сообщениями включает цепочку операций: sendMessage() обрабатывает ввод данных с валидацией содержимого, loadMessages() загружает историю через API с пагинацией, а displayMessages() оптимизирует рендеринг через виртуальный скролл. Редактирование реализовано через enableMessageEditing() с автоматическим сохранением версий, а deleteMessageById() обеспечивает мягкое удаление с архивированием.

\subsubsection{Управление коммуникациями}
	Работа с группами основана на loadGroups(), кэширующем список чатов при первом обращении. selectGroup() активирует выбранное пространство, подгружая участников через loadParticipants() с дебаунсингом запросов. Для личной переписки selectPrivateChat() инициирует защищенный канал, проверяя права доступа перед созданием сессии.
	
	\subsubsection{Работа с файлами}
		Система вложений использует updateFileIndicator() для отображения процесса загрузки, showPreviews() генерирует миниатюры изображений и иконки файлов. clearFiles() реализует отмену выбора с отзывом blob-объектов, а downloadAllAttachments() пакетирует файлы в zip-архив через Streams API.
		
		\subsection{Управление состоянием}
		Хранение данных организовано трехуровневой системой: 
		- SessionStorage сохраняет учетные данные и активный контекст чата 
		- Глобальные переменные отслеживают текущую группу (currentGroup) и приватную беседу (currentPrivateChat)
		- UI-состояние (открытые панели, модальные окна) управляется через наблюдаемые свойства DOM
		
		\subsection{Механизмы синхронизации}
		Обновление данных реализовано комбинацией трех подходов: 
		1. Фоновый опрос через checkForUpdates() (интервал 2.5 сек)
		2. Событийные триггеры на действия пользователя 
		3. Дебаунс-обертки (debouncedUpdate()) для ресурсоемких операций
		
		\subsection{Оптимизация и безопасность}
		Производительность обеспечена виртуализацией списков сообщений и пакетной обработкой DOM-изменений. Защита включает:
		- Санитизацию HTML через DOMPurify 
		- Валидацию полей ввода регулярными выражениями
		- Шифрование локальных данных с помощью Web Crypto API
		- Автоматический выход при невалидных токенах
		
		\subsection{Адаптивный интерфейс}
		Отзывчивый дизайн реализован через:
		- Медиа-запросы для 5 разрешений экрана
		- Динамическое изменение grid-раскладки
		- Прогрессивную загрузку изображений 
		- Гибкие контейнеры с clamp()-размерами
		
		\subsection{Обработка сбоев}
		Система ошибок использует:
		- Перехватчики Fetch-запросов 
		- Визуальные индикаторы с таймаутом
		- Экспоненциальную задержку повторов
		- Автоматический откат к последнему стабильному состоянию
		- Интеграцию с Sentry для критичных сбоев
