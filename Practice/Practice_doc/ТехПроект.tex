\section{Технический проект}
\subsection{Общая характеристика организации решения задачи}

Разрабатывается клиентская сторона веб-приложения для чат-платформы, обеспечивающая взаимодействие пользователя с серверной частью программы и базой данных. Клиентская сторона реализует следующие основные функции:
\begin{enumerate}
	\item Аутентификация и авторизация пользователей
	\item Обмен сообщениями в групповых чатах
	\item Личная переписка между пользователями
	\item Управление группами и правами участников
	\item Работа с вложениями и медиафайлами
\end{enumerate}

\subsection{Обоснование выбора технологии проектирования}

Для реализации клиентской стороны выбраны следующие технологии:

\subsubsection{Язык программирования JavaScript}

JavaScript выбран благодаря:
\begin{itemize}
	\item Поддержки в браузерах
	\item Простоте и читаемости кода
	\item Популярности и востребованности
	\item Кроссплатформенности
	\item Асинхронности и динамике
\end{itemize}

\subsubsection{Язык программирования CSS}

В качестве основы разметки используется CSS (Cascading Style Sheets) - основной язык стилизации, обеспечивающий гибкость и удобство работы с дизайном:
\begin{itemize}
	\item Легковесность - стили загружаются отдельно от HTML
	\item Простота развертывания - подключение к HTML через встроенные стили
	\item Совместимость - работает во всех современных браузерах
	\item Адаптивность - поддержка медиа-запросов для различных устройств
\end{itemize}

\subsubsection{Язык разметки HTML}

В качестве основы структуры веб-страницы используется HTML (HyperText Markup Language) - главный язык разметки, обеспечивающий логическую организацию контента и взаимодействие с CSS и JavaScript
\begin{itemize}
	\item Структурированность - позволяет логически организовывать элементы страницы
	\item Универсальность - работает во всех современных браузерах как для статических страниц, так и для динамических веб-приложений
	\item Кроссплатформенность - работает на любом устройстве (компьютерах, планшетах, смартфонах)
	\item Адаптивность - поддерживает медиа-запросы и гибкие макеты
\end{itemize}


\subsection{Клиентская часть мессенджера}


\subsubsection{Структура проекта}
\begin{xltabular}{\textwidth}{|l|X|}
	\caption{Файловая структура клиентской части}\label{tab:project-structure} \\ \hline
	\centrow{Тип} & \centrow{Описание} \\ \hline 
	\endfirsthead
	\continuecaption{Продолжение таблицы \ref{tab:project-structure}}
	\finishhead
	HTML-страницы & 
	\begin{itemize}[leftmargin=*,nosep]
		\item index.html — основной интерфейс мессенджера
		\item login.html — страница авторизации
		\item register.html — страница регистрации
		\item 404.html, 500.html — страницы ошибок
	\end{itemize} \\ \hline
	
	Стили & 
	\begin{itemize}[leftmargin=*,nosep]
		\item style.css — основные стили приложения
		\item login.css, register.css — стили форм авторизации
		\item error.css — стили страниц ошибок
	\end{itemize} \\ \hline
	
	JavaScript & 
	\begin{itemize}[leftmargin=*,nosep]
		\item app.js — основной клиентский код 
		\item login.js — обработка формы входа
		\item register.js — обработка формы регистрации
	\end{itemize} \\ \hline
	
\end{xltabular}

\subsection{Ключевые функции и методы}

\subsubsection{Управление интерфейсом}

\begin{itemize}
	\item \textbf{initUI()} - инициализация всех элементов интерфейса
	\item \textbf{toggleSidebar()} - открытие/закрытие боковых панелей
	\item \textbf{updateAuthButtons()} - обновление блока авторизации
	\item \textbf{showToast(message, type)} - отображение уведомлений
\end{itemize}

\subsubsection{Работа с сообщениями}

\begin{itemize}
	\item \textbf{sendMessage(e)} - обработка отправки сообщения
	\item \textbf{loadMessages()} - загрузка сообщений текущего чата
	\item \textbf{displayMessages(messages)} - отрисовка сообщений в чате
	\item \textbf{enableMessageEditing()} - включение режима редактирования
	\item \textbf{deleteMessageById()} - удаление сообщения
\end{itemize}

\subsubsection{Группы и чаты}

\begin{itemize}
	\item \textbf{loadGroups()} - загрузка списка групп
	\item \textbf{selectGroup()} - выбор активной группы
	\item \textbf{createGroup()} - создание новой группы
	\item \textbf{loadParticipants()} - загрузка участников группы
	\item \textbf{selectPrivateChat()} - выбор личного чата
\end{itemize}

\subsubsection{Работа с файлами}

\begin{itemize}
	\item \textbf{updateFileIndicator()} - отображение выбранных файлов
	\item \textbf{showPreviews(files)} - превью вложений перед отправкой
	\item \textbf{clearFiles()} - очистка выбранных файлов
	\item \textbf{downloadAllAttachments()} - скачивание всех вложений
\end{itemize}

\subsection{Обработка состояний}

Приложение использует несколько механизмов хранения состояния:

\begin{itemize}
	\item \textbf{SessionStorage}:
	\begin{itemize}
		\item Текущий пользователь (username)
		\item Выбранный чат/группа
		\item Временные метки последних сообщений
	\end{itemize}
	
	\item \textbf{Глобальные переменные}:
	\begin{itemize}
		\item Текущая группа (currentGroup)
		\item Текущий личный чат (currentPrivateChat)
		\item Временная метка последнего сообщения (lastTimestamp)
	\end{itemize}
	
	\item \textbf{UI состояние}:
	\begin{itemize}
		\item Открытые/закрытые панели
		\item Активные элементы интерфейса
		\item  Модальные окна
	\end{itemize}
	
\end{itemize}

\subsection{Механизмы обновления данных}

\begin{itemize}
	\item \textbf{Периодический опрос}:
	\begin{itemize}
		\item checkForUpdates() - проверка новых сообщений
		\item checkInterfaceUpdates() - обновление списков чатов
	\end{itemize}
	
	\item \textbf{Дебаунс}:
	\begin{itemize}
		\item debouncedUpdate() - ограничение частоты запросов
		\item debouncedLoadParticipants() - оптимизация загрузки участников
	\end{itemize}
	
	\item \textbf{Событийная модель}:
	\begin{itemize}
		\item Обработчики действий пользователя
		\item Реакция на изменения в DOM
	\end{itemize}
	
\end{itemize}

\subsection{Особенности реализации}

\subsubsection{Оптимизации производительности}

\begin{itemize}
	\item Виртуализация списка сообщений (рендеринг только видимых элементов)
	\item Кэширование данных на клиенте
	\item Пакетная обработка обновлений
	\item Оптимизированные селекторы DOM
\end{itemize}

\subsubsection{Безопасность}

\begin{itemize}
	\item Санитизация вводимых данных
	\item Обработка ошибок API
	\item Защита от XSS (экранирование вывода)
	\item Валидация перед отправкой на сервер
\end{itemize}

\subsubsection{Адаптивный дизайн}

\begin{itemize}
	\item Медиа-запросы для разных разрешений
	\item Динамическое изменение layout
	\item Оптимизированные формы ввода для мобильных устройств
	\item Гибкие контейнеры и элементы
\end{itemize}

\subsection{Обработка ошибок}

Клиентская часть реализует следующие механизмы обработки ошибок:

\begin{itemize}
	\item Перехват ошибок API-запросов
	\item Визуальное отображение ошибок пользователю
	\item Автоматические повторные попытки для неудачных запросов
	\item Восстановление состояния после ошибок
	\item Перенаправление на страницы ошибок (404, 500) при необходимости
\end{itemize}

