\section{Технический проект}
\subsection{Общая характеристика организации решения задачи}

Разрабатывается клиентская сторона веб-приложения для чат-платформы, обеспечивающая взаимодействие пользователя с серверной частью программы и базой данных. Клиентская сторона реализует следующие основные функции:
\begin{itemize}
	\item аутентификация и авторизация пользователей;
	\item обмен сообщениями в групповых чатах;
	\item личная переписка между пользователями;
	\item управление группами и правами участников;
	\item работа с вложениями и медиафайлами.
\end{itemize}

\subsection{Обоснование выбора технологии проектирования}

\subsubsection{Язык программирования JavaScript}

JavaScript выбран благодаря:
\begin{itemize}
	\item поддержки в браузерах;
	\item простоте и читаемости кода;
	\item популярности и востребованности;
	\item кроссплатформенности;
	\item асинхронности и динамике.
\end{itemize}

\subsubsection{Язык программирования CSS}

В качестве основы разметки используется CSS (Cascading Style Sheets) - основной язык стилизации, обеспечивающий гибкость и удобство работы с дизайном:
\begin{enumerate}
	\item Легковесность - стили загружаются отдельно от HTML.
	\item Простота развертывания - подключение к HTML через встроенные стили.
	\item Совместимость - работает во всех современных браузерах.
	\item Адаптивность - поддержка медиа-запросов для различных устройств.
\end{enumerate}

\subsubsection{Язык разметки HTML}

В качестве основы структуры веб-страницы используется HTML (HyperText Markup Language) - главный язык разметки, обеспечивающий логическую организацию контента и взаимодействие с CSS и JavaScript:
\begin{enumerate}
	\item Структурированность - позволяет логически организовывать элементы страницы
	\item Универсальность - работает во всех современных браузерах как для статических страниц, так и для динамических веб-приложений
	\item Кроссплатформенность - работает на любом устройстве (компьютерах, планшетах, смартфонах)
	\item Адаптивность - поддерживает медиа-запросы и гибкие макеты
\end{enumerate}


\subsection{Клиентская часть мессенджера}


\subsubsection{Структура проекта}  
\begin{xltabular}{\textwidth}{|l|X|}  
	\caption{Файловая структура клиентской части}\label{tab:project-structure} \\ \hline  
	\centrow{Тип} & \centrow{Описание} \\ \hline   
	\endfirsthead  
	\continuecaption{Продолжение таблицы \ref{tab:project-structure}}  
	\finishhead  
	
	HTML-страницы &  
	Основные страницы клиентской части включают:  
	\begin{itemize}[leftmargin=*,nosep]  
		\item index.html: основной интерфейс мессенджера;  
		\item login.html: страница авторизации;  
		\item register.html: страница регистрации;  
		\item 404.html, 500.html: страницы ошибок.  
	\end{itemize} \\ \hline  
	
	Стили &  
	Файлы стилевой разметки содержат:  
	\begin{itemize}[leftmargin=*,nosep]  
		\item style.css: основные стили приложения;  
		\item login.css, register.css: стили форм авторизации;  
		\item error.css: стили страниц ошибок.  
	\end{itemize} \\ \hline  
	
	JavaScript &  
	Клиентские модули представлены:  
	\begin{itemize}[leftmargin=*,nosep]  
		\item app.js: основной клиентский код;  
		\item login.js: обработка формы входа;  
		\item register.js: обработка формы регистрации.  
	\end{itemize} \\ \hline  
\end{xltabular}  

\subsection{Ключевые функциональные модули}

\subsubsection{Управление интерфейсом}
Система взаимодействия с пользователем реализуется через initUI(), инициализирующую все элементы при загрузке страницы. toggleSidebar() управляет видимостью боковых панелей, синхронизируя их состояние с SessionStorage. Динамическое обновление блока авторизации (updateAuthButtons()) отражает изменения статуса пользователя, а унифицированная система уведомлений (showToast()) использует анимированные всплывающие сообщения с цветовой индикацией типа события.

\subsubsection{Обработка сообщений}
Механизм работы с сообщениями включает цепочку операций: sendMessage() обрабатывает ввод данных с валидацией содержимого, loadMessages() загружает историю через API с пагинацией, а displayMessages() оптимизирует рендеринг через виртуальный скролл. Редактирование реализовано через enableMessageEditing() с автоматическим сохранением версий, а deleteMessageById() обеспечивает мягкое удаление с архивированием.

\subsubsection{Управление коммуникациями}
	Работа с группами основана на loadGroups(), кэширующем список чатов при первом обращении. selectGroup() активирует выбранное пространство, подгружая участников через loadParticipants() с дебаунсингом запросов. Для личной переписки selectPrivateChat() инициирует защищенный канал, проверяя права доступа перед созданием сессии.
	
	\subsubsection{Работа с файлами}
		Система вложений использует updateFileIndicator() для отображения процесса загрузки, showPreviews() генерирует миниатюры изображений и иконки файлов. clearFiles() реализует отмену выбора с отзывом blob-объектов, а downloadAllAttachments() пакетирует файлы в zip-архив через Streams API.
		
		\subsection{Управление состоянием}  
		Хранение данных организовано трехуровневой системой:  
		\begin{itemize}  
			\item sessionStorage сохраняет учётные данные и активный контекст чата;  
			\item глобальные переменные отслеживают текущую группу (currentGroup) и приватную беседу (currentPrivateChat);  
			\item UI-состояние (открытые панели, модальные окна) управляется через наблюдаемые свойства DOM.  
		\end{itemize}  
		
		\subsection{Механизмы синхронизации}  
		Обновление данных реализовано комбинацией трёх подходов:  
		\begin{itemize}  
			\item фоновый опрос через checkForUpdates() (интервал 2.5 сек);  
			\item событийные триггеры на действия пользователя;  
			\item дебаунс-обёртки (debouncedUpdate()) для ресурсоёмких операций.  
		\end{itemize}  
		
		\subsection{Оптимизация и безопасность}  
		Производительность и защита системы обеспечены:  
		\begin{itemize}  
			\item санитизация HTML через DOMPurify;  
			\item валидация полей ввода регулярными выражениями;  
			\item шифрование локальных данных с помощью Web Crypto API;  
			\item автоматический выход при невалидных токенах.  
		\end{itemize}  
		
		\subsection{Адаптивный интерфейс}  
		Отзывчивый дизайн реализован через:  
		\begin{itemize}  
			\item медиа-запросы для 5 разрешений экрана;  
			\item динамическое изменение grid-раскладки;  
			\item прогрессивную загрузку изображений;  
			\item гибкие контейнеры с clamp()-размерами.  
		\end{itemize}  
		
		\subsection{Обработка сбоев}  
		Система обработки ошибок включает:  
		\begin{itemize}  
			\item перехватчики Fetch-запросов;  
			\item визуальные индикаторы с таймаутом;  
			\item экспоненциальную задержку повторов;  
			\item автоматический откат к последнему стабильному состоянию;  
			\item интеграцию с Sentry для критичных сбоев.  
		\end{itemize}  
