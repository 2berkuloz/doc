\section{Анализ предметной области}
\subsection{Современные тенденции корпоративных коммуникаций}

Корпоративные мессенджеры - это специализированные платформы, ориентированные на оптимизацию рабочих процессов. В отличие от массовых сервисов (WhatsApp, Telegram), они предоставляют инструменты для структурированной коммуникации внутри организаций, интеграции с бизнес-приложениями и контроля данных.
Основными функциями корпоративного мессенджера являются:
\begin{itemize}
	\item обмен сообщениями и файлами;
	\item безопасность и контроль данных;
	\item организация коммуникации;
	\item групповые чаты сотрудников;
	\item личные чаты между сотрудниками;
	\item интеграция с другими сервисами и системами для управления задачами;
	\item управление доступом для пользователей (администратор, модератор, сотрудник, иной пользователь);
	\item кроссплатформенность;
	\item встроенная IP-телефония;
	\item таймер автоматического удаления сообщений и файлов.
\end{itemize}


На отечественном, так и на зарубежном рынке были успешно представлены такие решения:
\begin{enumerate}
	\item Microsoft Teams: глубокая интеграция с экосистемой Microsoft и поддержка многотысячных команд.
	\item Slack: гибкие настройки рабочих процессов через API и приложения, включая интеграцию с GitHub.
	\item Zulip: уникальная система потоков в чатах, которая упрощает отслеживание тематических обсуждений.
	\item СберЧаты (СберБизнес): решение с поддержкой E2E-шифрования и интеграцией с сервисами экосистемы Сбера.
	\item eXpress: классический мессенджер с расширенными возможностями для защищённой корпоративной коммуникации и командной работы.
\end{enumerate}

Корпоративные чаты, в отличие от обычных мессенджеров, ориентированы на рабочие процессы. Вы общаетесь преимущественно с коллегами, а если нужно пригласить к беседе клиента или подрядчика, то выдаете ему гостевой доступ или временную учетную запись. Такие корпоративные чаты позволяют лучше организовать процессы.

Рынок корпоративных мессенджеров предлагает различные модели установки — на своих серверах или в облаке компании. Это позволяет контролировать безопасность самостоятельно, выбирая вариант, который подходит больше всего.

Ситуация с зарубежными мессенджерами в России в 2025 году

Slack и MS Teams ушли с российского рынка, а западные мессенджеры блокируются.

Корпоративные мессенджеры 2025 года должны удовлетворять ключевым требованиям бизнеса:
\begin{enumerate}
	\item \textbf{Безопасность}: Шифрование данных, двухфакторная аутентификация, контроль доступа
	\item \textbf{Гибкость}: Поддержка групповых и личных чатов, системы уведомлений, интеграция с корпоративными сервисами
	\item \textbf{Кроссплатформенность}: Веб-интерфейс + мобильные приложения
	\item \textbf{Производительность}: Минимальные задержки при обмене сообщениями
\end{enumerate}

{Ключевые инновации проекта}
Гибкая система управления группами с ролевой моделью:
\begin{itemize}
	\item администратор;
	\item модератор;
	\item сотрудник;
	\item клиент.
\end{itemize}
		

