\section{Техническое задание}
\subsection{Основание для разработки}

Основанием для разработки является задание на выпускную квалификационную работу бакалавра "<Бизнес-проект: Закрытый корпоративный мессенджер">.

\subsection{Цель и назначение разработки}

Разработка защищенного корпоративного мессенджера для внутреннего обмена сообщениями сотрудников компании с функциями:
\begin{itemize}
	\item шифрование передаваемых сообщений;
	\item групповое взаимодействие с ролевым доступом;
	\item хранение истории переписки.
\end{itemize}

	
\subsection{Функционал мессенджера}

\subsubsection{Аутентификация}  
Основные модули аутентификации включают:  
\begin{itemize}  
	\item регистрация по корпоративному логину;  
	\item авторизация с проверкой учётных данных.  
\end{itemize}  

\subsubsection{Чаты и сообщения}  
Функционал чатов и сообщений предусматривает:  
\begin{itemize}  
	\item личные сообщения между сотрудниками;  
	\item групповые чаты с управлением участниками;  
	\item отправка текстовых сообщений и файлов;  
	\item поиск по истории сообщений.  
\end{itemize}  

\subsubsection{Управление группами}  
Возможности управления группами состоят в следующем:  
\begin{itemize}  
	\item создание/удаление чатов;  
	\item добавление/исключение участников;  
	\item назначение ролей (владелец, администратор, участник).  
\end{itemize}  

\subsection{Роли пользователей}

\begin{xltabular}{\textwidth}{|l|X|}
	\caption{Роли пользователей}\label{tab:roles} \\ \hline
	\centrow Роль & \centrow Права \\ \hline
	\endfirsthead
	Обычный пользователь & 
	\begin{itemize}
		\item Личная переписка
		\item Участие в групповых чатах
		\item Отправка сообщений и файлов
	\end{itemize} \\ \hline
	Администратор чата & 
	\begin{itemize}
		\item Все права обычного пользователя
		\item Добавление/удаление участников
		\item Переименование чата
	\end{itemize} \\ \hline
	Владелец &
	\begin{itemize}
		Все права доступа
	\end{itemize} \\ \hline
\end{xltabular}

\subsubsection{Тестирование и отладка}  
Процесс тестирования и отладки включает:  
\begin{itemize}  
	\item проверка безопасности;  
	\item функциональное тестирование интерфейсов;  
	\item тестирование удобства использования.  
\end{itemize}  

\subsubsection {Тестирование и отладка:}
\begin{itemize}
    \item Проверка безопасности
	\item Функциональное тестирование интерфейсов
	\item Тестирование удобства использования
\end{itemize}
\newpage
\subsection{Требования к интерфейсу}

\subsubsection{Основные элементы интерфейса}

\begin{figure}[ht]
	\centering
	\includegraphics[width=0.8\linewidth]{"images/UI макет"}
	\caption{Схема интерфейса мессенджера}
	\label{fig:ui-main}
\end{figure}

\begin{figure}[ht]
	\centering
	\includegraphics[width=0.8\linewidth]{"images/UI макет регистрации"}
	\caption{Схема интерфейса формы регистрации}
	\label{fig:ui-reg}
\end{figure}

\begin{figure}[ht]
	\centering
	\includegraphics[width=0.8\linewidth]{"images/UI макет авторизации"}
	\caption{Схема интерфейса формы авторизации}
	\label{fig:ui-auth}
\end{figure}

\subsection{Сценарии использования}

\subsubsection{Регистрация нового пользователя}  
Процедура регистрации нового пользователя включает следующие шаги:  
\begin{itemize}  
	\item пользователь нажимает кнопку "Зарегистрироваться" на форме авторизации;  
	\item система отображает форму регистрации (рис. \ref{fig:ui-reg}) с полями:  
	\begin{itemize}  
		\item корпоративная почта (логин);  
		\item пароль (с требованиями сложности);  
		\item подтверждение пароля;  
	\end{itemize}  
	\item пользователь заполняет все обязательные поля;  
	\item пользователь нажимает кнопку "Зарегистрироваться";  
	\item система проверяет данные:  
	\begin{itemize}  
		\item при корректных данных:   
			\item создаёт новую учётную запись;  
			\item отправляет подтверждение на корпоративную почту;  
			\item перенаправляет на форму авторизации;  
			\item выводит сообщение "Регистрация успешно завершена";   
		\item при ошибках:  
			\item выделяет проблемные поля;  
			\item показывает соответствующие сообщения об ошибках:  
				\item "Пароль должен содержать не менее 8 символов";  
				\item "Пароли не совпадают";  
				\item "Учётная запись с таким именем уже существует";   
	\item после успешной регистрации администратор получает уведомление о новом пользователе для подтверждения корпоративного доступа.  
\end{itemize}  

\subsubsection{Авторизация пользователя}  
Процесс авторизации выполняется по схеме:  
\begin{itemize}  
	\item пользователь открывает веб-интерфейс мессенджера;  
	\item система отображает форму авторизации (рис. \ref{fig:ui-auth});  
	\item пользователь вводит корпоративный логин и пароль;  
	\item пользователь нажимает кнопку "Войти";  
	\item система проверяет учётные данные:  
		\item при успехе — загружает основной интерфейс (рис. \ref{fig:ui-main});  
		\item при ошибке — показывает сообщение "Неверный логин или пароль";   
	\item при нажатии "Зарегистрироваться" система перенаправляет на форму регистрации (рис. \ref{fig:ui-reg}).  
\end{itemize}  

\subsubsection{Создание группового чата}  
Алгоритм создания группового чата:  
\begin{itemize}  
	\item пользователь нажимает кнопку "+" (Создать чат) на левой панели;  
	\item система отображает диалоговое окно:  
		\item поле ввода названия чата;  
		\item список доступных сотрудников;  
		\item чекбоксы для выбора участников;   
	\item пользователь вводит название чата;  
	\item пользователь отмечает нужных участников;  
	\item пользователь нажимает кнопку "Создать";  
	\item система:    
		\item создаёт новый чат;  
		\item добавляет выбранных участников;  
		\item отображает новый чат в списке.  
\end{itemize}  

\subsubsection{Отправка сообщений}  
Логика отправки сообщений:  
\begin{itemize}  
	\item пользователь выбирает чат из списка;  
	\item система загружает историю переписки;  
	\item пользователь вводит текст в нижнее поле ввода;  
	\item пользователь может:   
		\item нажать кнопку "Отправить" (или Enter);  
		\item нажать кнопку "Прикрепить файл" и выбрать файл;    
	\item система:   
		\item шифрует и отправляет сообщение;  
		\item отображает сообщение в истории чата;  
		\item для файлов — показывает превью и название.   
\end{itemize}  

\subsubsection{Управление участниками группы (для администраторов)}  
Процедура управления участниками:  
\begin{itemize}  
	\item пользователь открывает групповой чат;  
	\item пользователь нажимает иконку "Управление чатом" в заголовке;  
	\item система отображает меню:  
		\item "Добавить участника";  
		\item "Исключить участника";  
		\item "Назначить администратора";   
	\item при выборе "Добавить участника":  
		\item открывается список сотрудников;  
		\item администратор выбирает сотрудников;  
		\item нажимает "Добавить";  
		\item система присылает уведомление новым участникам;  
	\item при выборе "Исключить участника":   
		\item открывается список текущих участников;  
		\item администратор выбирает участника;  
		\item нажимает "Исключить";  
		\item система удаляет участника из чата.  
\end{itemize}  

\subsection{Требования к оформлению документации}

Документация должна соответствовать ГОСТ 19.102-77 и ГОСТ 34.601-90. Единая система программной документации.